\section{Nombre: Huenupan}  \label{per:huenupan}
\subsection{Descripción:} 
Hombre de mediana edad y corpulencia delgada. De tez morena oscura y cabello largo. Usa un penacho para denotar su posición como cacique. Viste una capa de algodón fino que cuelga de sus espalada de manera recta y con un taparrabos de manta de colores. Porta dos brazaletes de diferentes acabados, uno en cada brazo. Y usa unas sandalias sencillas del color de su capa.     
\\
\par
De actitud pasiva y obediente; Huenupan carece de las actitudes para ser un lider por lo que se limita a obdecer las ordenes del imperio. 
\subsection{Status:}
\begin{itemize}
		\item Personaje no jugable.
	\end{itemize}
Es uno de los fantasmas del pasado que atormentan a Malinalli.
\subsection{Imagen}
	Sin imagen. 
\subsection{Concepto:}
\begin{itemize}
	\item \textbf{Historia antes del juego:}
	Originario de una familia noble del  norte de Veracruz, logra ascender hasta cacique debido a su ciega lealtad al imperio. Acepta el puesto sin conocer las necesidades de la población, siendo su prioridad garantizar el control del imperio pero no el bienestar de su pueblo.  
	\item \textbf{Historia durante el juego:}
	Su presencia se conoce por los recuerdo de Malinalli, o tiene participación directa en la historia que sucede en el Mictlán.
	\item \textbf{Relaciones:}
	\begin{itemize}
		\item \textbf{Cimatl:} Esposa de Huenupan. Acepta casarse con ella para tomar el puesto de cacique. Debido a su personalidad pasiva, usualmente termina siendo manipulado por su esposa para que haga lo que ella desea (ver aparatado \ref{per:cimatl}). 
		\item \textbf{Malinalli:} No guarda ningún cariño por ella (ver aparatado \ref{per:malinalli}). 
	\end{itemize}                     
\end{itemize}

\subsection{Encuentro:}
Su primera aparición es el la cinemática 25 (ver aparatado \ref{Cin:Cinematica25}).
\subsection{Habilidades:}
Es un guerrero de pocas habilidades. 
\subsection{Armas:}
Sin armas.
\subsection{Ítems:}
Sin ítems.
\subsection{Bloques de animación}
\begin{itemize}
	\item Animación caminar.
	\item Animación normal.
\end{itemize}