\section{Cinemática 32. Guarida de Itztlacoliuhqui. int/tarde.} \label{Cin:Cinematica32}
 \textsc{Personajes}:
 \begin{itemize}
 \item Xólotl.
 \item Malinalli.
 \item Itztlacoliuhqui.
 \end{itemize}
\textit{Itztlacoliuhqui está sentado en el centro de la habitación, cuando ve a Malinalli y a Xólotl entrar mueve las manos y les lanza flechas. Malinalli bloquea las flechas con el poder de la caracola. Itztlacoliuhqui salta de su asiento y aterriza donde Xólotl y Malinalli se encuentran rompiendo la defensa de Malinalli, seguido de esto lanza a Malinalli hasta el otro lado de la habitación y toma a Xólotl del cuello.}
\begin{center}
\textsc{\underline{Itztlacoliuhqui}}
\\
\par
Dame un motivo para no romperte el cuello en estos momentos.
\\
\par
\textsc{\underline{Xólotl}}
\\
\par
Después de todo lo que he hecho, estoy completamente seguro que Tezcatlipoca va a quererme vivo para castigarme él mismo. No querrás quitarle el gusto ¿O sí?  (Itztlacoliuhqui invoca más flechas) De acuerdo, mal motivo. Deseo una alianza. Venga, Itztlacoliuhqui. Tú y yo somos bastante parecidos. Ambos nos rebelamos contra el resto al inicio del quinto sol. Las cicatrices que te hizo Tonatiuh, puedes regresárselas sin más.
\\
\par
\textsc{\underline{Itztlacoliuhqui}}
\\
\par
No soy quien tu recuerdas. Aquello que buscas no me interesa. Solo hay algo que tienes que me importa y no necesito una alianza para recuperarla. (Malinalli se levanta y con la energía de la caracola ataca a Itztlacoliuhqui. Itztlacoliuhqui suelta a Xólotl. Xólotl se pone atrás de Malinalli).
\\
\par
\textsc{\underline{Malinalli}}
\\
\par
Él no va a escucharte. Solo hay dos maneras de que se reencuentre con su amada y no pienso caer después de estar tan cerca de recuperar todo.
\\
\par
\textsc{\underline{Itztlacoliuhqui}}
\\
\par
(Se reincorpora del ataque) ¿Cómo te atreves a usar su energía? He cambiado de opinión. Primero te destruiré a ti y después a Xólotl (Itztlacoliuhqui empieza a aumentar su tamaño hasta volverse un monstruo de grandes proporciones. Con su cambio de forma su guarida se destruye).
\end{center}