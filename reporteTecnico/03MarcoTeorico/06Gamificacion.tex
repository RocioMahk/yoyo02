\section{Gamificación}
La gamificación es el uso de las mecánicas de juego en entornos ajenos al juego, según el término anglosajon definido por Sebastian Deterding (Diseñador/investigador del diseño de juego para el florecimiento humano) \cite{gameDef}. Es decir, que cualquier tema o asunto a tratar puede pasar por un proceso para convertirse en un juego. Y deben de cumplir con características específicas según el profesor Santiago Moll\cite{gameficacion}, estas son: mecánicas o reglas, dinámicas de juego, y componentes. También describe que clase de jugadores existen, el proceso que debe de llevar un tema a gamificar y la finalidad que debe cumplir. Todo esto se muestra a continuación.
\\[1pt]

\subsection{Característcas}
Las características a presentar son una guía para realizar un juego y no necesariamente se debe cumplir con todas y cada una de ellas. Sin embargo estas características ayudan en gran medida al entretenimiento del jugador. Recordemos que las características a presentar incluyen las mecánicas o reglas, dinámicas de juego y componentes.
\\[1pt]
 
\textbf{Mecánicas o reglas}
\\[1pt]
Son las normas de funcionamiento que permiten se adquiera un compromiso del jugador con el juego. Mantienen al jugador en constante actividad y le permite ver los límites que existen en el juego.
\\[1pt]

\begin{itemize}
	\item Colección: Logros y recompensas que consigue el jugador.
	\item Puntos: Para motivación y conteo de realizar una tarea por el jugador.
	\item Ranking: Clasificación o comparación entre jugadores.
	\item Nivel: Reflejan el progreso del jugador.
	\item Progresión: Consiste en completar el 100\% de la actividad encomendada al jugador.	
\end{itemize}

\textbf{Dinámicas de juego}
\\[1pt]
Motivan y despiertan el interés del jugador de realizar una actividad dentro del juego. Aunque no es necesario cumplir con este requisito para un juego son clave para mantener jugando a la persona.  

\begin{itemize}
	\item Recompensa: Premio por realizar alguna actividad en el juego.
	\item Competición: Deseo de estar en una determinada posición o grado en el juego entre los participantes.
	\item Cooperativismo: Otra forma de competir pero en un grupo de jugadores con un mismo fin o meta del juego.
	\item Solidaridad: Se fomenta la ayuda entre jugadores y debe ser de manera altruista.
\end{itemize}

\textbf{Componentes}
\\[1pt]
Los componentes se encargan de personalmente darle a cada jugador su estado en el juego. Así son identificados por los demás participantes fácilmente en las actividades que destacan o han realizado con exito. También cumplen con la parteen la que el jugador puede proyectarse dentro juego.
\begin{itemize}
	\item Logros: Visualizan el alcance del jugador de un objetivo del juego.
	\item Avatares: Representación gráfica del jugador.
	\item Medallas: Insignia o distintivo del jugador que ha ganado.
	\item Desbloqueo: Permiten avanzar en las actividades del juego gracias a actividades previas hechas por el jugador.
	\item Regalos: Un presente por la realización correcta de un reto por el jugador	.
\end{itemize}

\subsection{Tipos de jugadores}
Para saber que elementos debe llevar el juego a realizar debe conocerse los tipos de jugadores que existen. También se debe de saberse las motivaciones que los impulsan a seguir jugando. A continuación se muestran cuatro identificados.
\begin{itemize}
	\item Triunfador: Su finalidad es la consecución de logros y retos.
	\item Social: Le encanta interactuar y socializarse con el resto de compañeros.
	\item Explorador: Tiene tendencia a descubrir aquello desconocido.
	\item Competidor: Su finalidad es demostrar su superioridad frente a los demás.
\end{itemize}


\subsection{Proceso}
Ahora se seguirá los pasos para convertir un tema a un juego. Conociendo ya los elementos que disponemos y haber identificado al tipo de jugador que queremos, podemos convertir nuestro tema o actividad en un juego. 
\begin{itemize}
	\item Viabilidad: Determinar si el contenido que se quiere enseñar es jugable.
	\item Objetivos: Definir los objetivos del juego.
	\item Motivación: Valorar la predisposición y el perfil de jugadores.
	\item Implementación: Relación entre el juego y contenido a enseñar.
	\item Resultados: Evaluación de la actividad en el juego.
\end{itemize}


\subsection{Finalidad}
En cada juego se debe tener claro lo que quiere lograrse, es decir la finalidad del juego. Debe de determinarse que se desea lograr con el jugador.
\begin{itemize}
	\item Fidelización: Establecer un vínculo del contenido del juego con el jugador.
	\item Motivación: Crear una herramienta contra el aburrimiento del contenido a tratar.
	\item Optimización: Recompensar al jugador en aquellas tareas en las que no tiene previsto ningún incentivo.
\end{itemize}
