\section{Cultura.}\label{cultura}
	En esta sección se hablara de la cultura, primeramente definiendola, para después 
	mencionar sus principales características, los tipos de cultura que hay, la 
	importancia de la cultura y el impacto que han tenido los videojuegos en la misma.
	\subsection{¿Qué es la cultura?}\label{CulturaDef}
	La Organización de las Naciones Unidas para la Educación, la Ciencia y la 
	Cultura (UNESCO) define la cultura como: "El conjunto de los rasgos distintivos, 
	espirituales y materiales, intelectuales y afectivos que caracterizan a una 
	sociedad o un grupo social. La cultura engloba, además de las artes y las letras, 
	los modos de vida, los derechos fundamentales al ser humano, los sistemas de 
	valores, las tradiciones y las creencias y que la cultura da al hombre la capacidad 
	de reflexionar sobre sí mismo. Es ella la que hace de nosotros seres específicamente 
	humanos, racionales, críticos y éticamente comprometidos. A través de ella discernimos 
	los valores y efectuamos opciones. A través de ella el hombre se expresa, toma 
	conciencia de sí mismo, se reconoce como un proyecto inacabado, pone en cuestión 
	sus propias realizaciones, busca incansablemente nuevas significaciones, y crea 
	obras que lo trascienden"\cite{RefCultura}. Bajo esta definición se puede entender 
	a la cultura como un construcción humana que le da identidad a los individuos. 
	
	\subsection{Características de la cultura}\label{CulturaCaract}
	Con base en Puja Modal la cultura esta compuesta de las siguientes 
	caracteristicas\cite{RefculturaCarac}
	\begin{itemize}
		\item \textbf{La cultura es adquirida:} La cultura se aprende no es una 
		característica biológica inherente al nacer. 
		\item \textbf{La cultura es social:} La cultura se adquiere como producto de 
		las interacciones humanas. Sin una sociedad no puede existir una cultura.
		\item \textbf{La cultura es transmisiva:} La cultura se transmite de transmite 
		entre individuos, tiene un flujo dinámico y nunca permanece constante.
		\item \textbf{La cultura llena necesidades:} La cultura puede llenar diferentes 
		necesidades humanas como la moral, la solidaridad y la convivencia.
		\item \textbf{La cultura es compartida:} La cultura no es una posesión de un solo 
		individuo sino es algo que comparte una gran mayoría de una población en un espacio 
		determinado.
		\item \textbf{La cultura es idealista:} La cultura conglomera las ideas, 
		valores y normas del grupo dominante de una sociedad y los maneja como si todos 
		tuvieran los mismo valores e ideas. 
		\item \textbf{La cultura es acumulativa:} La cultura no se crea en cortos periodos 
		de tiempo sino es la suma de los ideales, creencias y normas que generacionalmente 
		se van creando.
		\item \textbf{La cultura es adaptable:} La cultura se adapta a diferentes 
		cambios y se modifica.
		\item \textbf{La cultura es variable:} No es absoluta, cada sociedad tiene su 
		propia cultura.
		\item \textbf{La cultura es organizada:} La cultura esta organizada por 
		diferentes conjuntos de culturas; estos se unen de manera ordenada para formar 
		un todo, la cultura de la sociedad. 
		\item \textbf{La cultura es comunicativa:}La cultura se basa en símbolos y en 
		como estos símbolos son comunicados entre los individuos de una sociedad. 
	\end{itemize}
	
	\subsection{Clasificación de la cultura.}\label{CulturaClasi}
	Existen diferentes tipos de de clasificación de la cultura pero en esta sección solo 
	se hace habla de la clasificación por definición,ya que esta clasificación es la que 
	aborda el tipo de cultura sobre la que trabaja el presente trabajo terminal. A 
	continuación se presentan los tipos de cultura según la clasificación por definición:
	\begin{itemize}
		\item \textbf{Tópica:} Esta clasificación consiste en una lista de tópicos o 
		categorías, tales como organización social, religión, seguridad, empleo, economía, 
		etc \cite{RefculturaClasificacion}.
		\item \textbf{Histórica:} Esta clasificación hace referencia a la herencia social. 
		Es la relación que tiene la sociedad con su pasado\cite{RefculturaClasificacionEl}. 
		\item \textbf{Cultura mental:} Este tipo de cultura engloba todos aquellos hábitos 
		o costumbres que diferencian a un individuo o un conjunto de individuos del resto. 
		Este tipo de cultura se puede entender como la idiosincrasia de una población
		\cite{RefculturaClasificacion}.
		\item \textbf{Cultura estructura:} Es el conjunto de símbolos, valores, creencias 
		y conductas reglamentadas y relacionados entre sí\cite{RefculturaClasificacionEl}. 
		\item \textbf{Cultura simbólica:} Conforma todas aquellas reglas, canales  modos 
		de comunicación que existen entre los individuos de una 
		sociedad\cite{RefculturaClasificacion}.

	\end{itemize}
	\subsection{Importancia de la cultura.}\label{CulturaImpo}
	La cultura es importante ya que ella:
	\begin{itemize}
		\item Determina la estructura del pensamiento, lo que influye en las percepciones, 
		los valores y el comportamiento \cite{RefImporCul}.
		\item Permite la construcción de piezas artísticas e históricas que sirven como 
		testimonio del pasado \cite{RefImpoCulAr}.		
		\item Da unidad y sentido de pertenencia\cite{RefImpoUnidad}.
		\item Permite la convivencia entre individuos \cite{RefImporCul}.
		\item Regula el comportamiento humano\cite{RefImporCul}.
		\item Permiten el crecimiento y recreación del individuo\cite{RefImpoCulAr}.
	\end{itemize}
	 

	 