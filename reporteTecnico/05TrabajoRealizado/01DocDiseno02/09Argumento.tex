\subsection{Argumento del videojuego}
Huddle maneja un apartado llamado \textbf{Guión} para documentar de manera detallada 
la historia del videojuego. No obstante, al igual que como sucede con algunos de 
los apartados ya descritos, Huddle no proporciona una plantilla para documentarlo. 
Ante la falta de una plantilla para documentar el argumento y ante la existencia 
de cinemáticas dentro de los niveles y fuera de estos, se decidió que el argumento 
del juego se documentaría como una animación. Contando así con tres guiones:
	\begin{itemize}
		\item \textbf{Guión literario}:Este guión es parecido a un guión teatral. 
		Por medio de escenas va desarrollando la historia, mostrando la secuencia 
		de diálogos que entablan los personajes participes en el argumento. Para 
		el caso particular del videojuego Yolotl, las escenas recibe el nombre de 
		cinemáticas. Cada cinemática se documentara bajo la siguiente plantilla:
			\begin{itemize}
				\item Numero de la cinemática seguido del nombre la locación donde 
				acontece ésta; en caso de que la escena suceda en el interior de 
				alguna edificación se pondrá seguido del nombre de la locación el 
				prefijo int, en caso de suceder en el exterior se colocará el 
				prefijo ext.
				\item Relación con los nombres de todos los personajes que participan 
				en la escena.
				\item Breve descripción de la locación.
				\item Secuencia de diálogos. Cada diálogo va precedido por el nombre 
				de personaje que lo dice, el nombre del personaje debe de ir en 
				mayúsculas y subrayado. 
			\end{itemize}
			\item Storyboard: El storyboard es una secuencia de imágenes que narran 
			de manera visual la historia. Cada imagen va comentada de tecnicismos 
			que faciliten la descripción de acciones (tales como el desplazamiento 
			de la cámara, movimientos de personajes, intención del personaje en 
			decir un diálogo, etc.) \cite{RefStoyBoard}.  			 
	\end{itemize}