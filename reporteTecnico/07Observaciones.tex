\chapter{Reflexiones}

% punto final del informe: "como ha sido mostrado...."

Como se vió durante el avance del proyecto, 
un videojuego educativo es una gran herramienta de ayuda para la enseñanza y el aprendizaje. Este es un medio controlado con normas establecidas, lo que facilita en control de información y en que modo debe aprenderse.

La cultura puede ser vista como entretenida gracias al juego. Se puede combinar de manera digerible por el jugador los componentes históricos y de juego. Y este tipo de combinación en material cultural con un videjuego es bien recibido por casi todos los diferentes tipos de personas.

La intención es que con los incentivos correctos para grupos específicos de personas, en este caso un videojuego, se pueda motivar a la sociedad a que se interese y aprenda conocimiento cultural. 
